%&context
\setuppapersize[letter][letter]
\setuplayout[topspace=0.75in, backspace=0.75in, height=middle, width=middle]
\setupwhitespace[medium]
\setuppagenumbering[location=footer]
\setupheader[state=none]
\usemodule[tikz]
\usetikzlibrary[graphs, graphdrawing, quotes]
\usegdlibrary[trees]
\usemodule[simplefonts]
\setmainfont[gentiumplus]

\starttext

\title{Design of a Selfing Simulator, {\sc SelfingSim}}

\startsection[title={Introduction}]
This document describes the design and implementation of {\sc SelfingSim},
a forward-in-time population genetics simulator with selfing
under pure-hermaphroditism, androdioecy, or gynodioecy.
\stopsection

\startsection[title={Design}]
{\sc SelfingSim} simulates a constant-size population without
generation overlap.
It uses the infinite-alleles model of mutation.
A single simulation consists of three phases:
initialization, iteration, and termination.
Much of simulation parameters can be set easily in a JSON-formatted input file.

\startsubsection[title={Initialization}]
At the beginning of a simulation, {\sc SelfingSim} initializes a population.
The first step is to set up a population of \m {N} reproducing orgniams,
each of which has \m {L} autosomal loci.
All organisms are hermaphroditic under pure-hermaphroditism,
but only \m {N_h} organisms are hermaprhoditic under andro- and gynodioecy.
The rest of the population are either males (androdioecy) or females (gynodioecy).

The simulator then initializes genotype.
Four methods of initializing genotypes are currently supported:
\quotation {\tt monomorphic}, \quotation {\tt unique}, \quotation {\tt count},
and \quotation {\tt frequency}.
All these methods work on a gene at a time.

The first two methods are simple.
All genes share a common allele in the \quotation {\tt monomorphic} method,
and no pair of genes share a common allele in the \quotation {\tt unique} method.
Under the \quotation {\tt count} method, a state of a gene is uniformly sampled from
\m {M} alleles.
Finally, \quotation {\tt frequency} method again randomly draws the state
from \m {M} alleles
similar to \quotation {\tt count}, but a frequency of each allele can be specified.

Preliminary simulations have found that the \quotation {\tt monomorphic} method
caused a population to slowly apporach mutation-drift equilibrium.
Furthermore, simulations with lower mutation rates took longer to accumulate
standing genetic variation than with higher mutation rates.
On the contrary, a population apporached the equilibrium at much faster rate in
the \quotation {\tt unique} method.
Its approach was mainly governed by efficiency of genetic drift to eliminate
excessive standing genetic variation from the population.
Based on these findings, the \quotation {\tt unique} method has been used
throughout my simulations.
The other two methods have not been explored.
\stopsubsection

\startsubsection[title={Iteration}]
After a population is initialized,
{\sc SelfignSim} starts simulating evolution of the population
for \m {T_{max}} generations.
Optionally, \m {T_B} generations of burn-in can precede the main
iterations.
These iterations use the exact same sequence of events at the exact
same rate.

One generation in a simulation consists of two events:
mutation and mating.
Mutations are introduced in a population at the beginning of an interation.
For each gene in each organisms, a test is performed to determine if the
gene is mutated using locus-specific mutation rates \m {M_i} for locus \m {i}.
If a mutation occurs in a gene, a new allele is assigned to the gene.
This new allele has not been present throughtout the entire
history of the population.

After mutation, the simulater performs mating.
At the end of mating, there are \m {N} newly generated organisms,
who survive and participate in the next round of mating.
Moreover, number of hermaphrodites \m {N_h} is likewise fixed
across generations under andro- and gynodioecy.

For each of three mating models, pure-hermaphroditism, andro-, and
gynodioecy, there are two ways to specify mating related parameters.
The first way is to supply fundamental parameters.
The other way is to specify compound parameters.

Depending on which set of parameters are specified, {\sc SelfingSim}
proceeds to simulate mating in slightly differently.
Following subsections describe model-specific mating in more detail.

\startsubsubsection[title={Pure-Hermaphroditism}]
Under pure hermaphroditism, all offspring are hermaphrodite capable of selfing and
outcrossing.

\startsubsubsubsection[title={Fundamental parameters}]
When a user specified fundamental parameters, the first event is to determine
if an offspring is reproduced by a selfing parent or a pair of parents.
An offspring has a selfing parent with probability \m {\tilde{s}}.
Then, a uniparental offspring survives until its maturity with probability
\m {\tau}, and a biparental offspring always survives.
When an offspring does not survive, a new offspring is drawn by repeating this
process from the beginning.

\stopsubsubsubsection

\startsubsubsubsection[title={Compound parameters}]

When a compound parameter \m {s^*} are supplied, an offspring is uniparental
with probability

\startformula
s^* = s_H = \frac{ \tilde {s} \tau }{ \tilde {s} \tau + 1 - \tilde {s} }.
\stopformula
Otherwise, a offspring is biparental.

\stopsubsubsubsection
\stopsubsubsection

\startsubsubsection[title={Androdioecy}]
Under androdioecy, there are males and hermaprhodites.
Males are incapable of selfing, and they require eggs from hermaprhodites
to reproduce.
Hermaprhodites are capable of selfing and outcrossing.
However, they can't utilize sperm/pollen from other hermaphrodites.

\startsubsubsubsection[title={Fundamental parameters}]
Given fundamental parameters \m {s^*} and \m {\tau}, the same set of parameters
as the parameters under pure-hermaphroditism, the first step is to determine
if an offspring is uniparental with probability \m {s^*}.
If uniparental, an offspring is necessarily hermaphroditic, which survive until
its maturity with probability \m {\tau}.
If biparental (with probability \m {1-s^*}), it has a male seed/pollen parent and
a hermaprhoditic egg parent.
All biparental offspring always survive.
If a uniparental offspring dies, its place is replaced by another uniprental
offspring.
This replacing offspring is also drawn using the same steps.

\stopsubsubsubsection

\startsubsubsubsection[title={Compound parameters}]
Alternatively, a compound parameter \m {s^*} is specified, this can be used in
mating as is in pure-hermaphroditism.
An offspring is uniparental with probability
\startformula
s^* = s_A = \frac{ \tilde {s} \tau }{ \tilde {s} \tau + 1 - \tilde {s} },
\stopformula
where its hermaphroditic parent is drawn randomly from \m {N_h} hermaphrodites.
Otherwise, it is biparental with one parent each drawn from male and her
chosen from \m {N - N_h} males and \m {N_h} hermaphrodites.

\stopsubsubsubsection

\stopsubsubsection

\startsubsubsection[title={Gynodioecy}]
A gynodioecious population consists of two groups of organism:
hermaphrodites and females.
Hermaphrodites can generate sperms/pollens and eggs, and they can reproduce with
itself, another hermaphrodite, and a female.
Females can generate only egg, and they need hermaphrodites to provide
sperms/pollens.

\startsubsubsubsection[title={Fundamental parameters}]
With fundamental parameters, the first step is to decide if a seed-parent of an
offspring is female or hermaphroditic.
A seed-parent is female with probability \m {N_f \sigma/(N_f \sigma + N_h)},
where \m {\sigma} is the number of seed a female produces relative to
a hermaphrodite.
An offspring of female parent is always biparenal, and they always survive.
On the other hand, if an offspring has a hermaphroditic seed-parent,
the offspring is uniparental with probability \m {a} and biparental otherwise.
When uniparental, an offspring dies with probability \m {\tau} before its maturity.
If a uniparental offspring dies, a replacement offspring is selected following
the same steps.

\stopsubsubsubsection

\startsubsubsubsection[title={Compound parameters}]
Again, compopund parameters can be supplied in place of fundamental parameters.
When compound parameters \m {s^*} and \m {H} are supplied, an offspring is
uniparental with probability
\startformula
s^* = s_G = \frac{ \tau N_h a }{ \tau N_h a + N_h (1 - a) + N_f \sigma }.
\stopformula
Otherwise, an offspring is biparental.
If an offspring is biparental, its seed-parent is hermaphroditic with
probability
\startformula
H = \frac{ N_h (1 - a) }{ N_h (1 - a) + N_f \sigma },
\stopformula
and female with probability \m {1-H}.

\stopsubsubsubsection

\stopsubsubsection

\startsubsubsection[title={Formation of an offspring}]
Each generation has exactly \m {N_h} hermaphrodites under andro- and
gynodioecy.
Genotype of an offspring is determined by the standard autosomal transimssion
from its parent(s).
If organisms have more than one loci, recombination rates between
adjacent loci \m {r} are identical and user-specified.
In addition to genetic transmission, {\sc SelfingSim} keeps track of the
number of generations since the last non-selfing event.
\stopsubsubsection

\startsubsubsection[title={Saving state of a population}]
After mating and reproduction, the simulator can optionally save the state
of an entire population.
Stored information are state of genes and the number of generations since
the last outcrossing event.
A user can specify frequency of this operation.
\stopsubsubsection

\stopsubsection

\startsubsection[title={Termination}, reference=sec:term]
During the termination phase of a simulation, the state of the
last generation is stored in file.
The information saved during this stage is identical to the ones stored
during the simulation (genotype and time since the last outcrossing event).

\stopsubsection

\stopsection

% \startsection[title={Choosing Parent(s) under Gynodioecy}]
% There are several ways to choose parents of an offspring under gynodioecy.
% This section shows equivalence between those two parent-choosing methods.
%
% In order to choose parent(s) under gynodioecy, three events have to occur
% in sequence.
% These events are a) the number of parents, b) sex of seed parent, and c)
% survival of an offspring until its maturity.
% Two methods differ in whether the number of parents is decided before sex of seed parents or not.
%
% Le \m {\{\tau'_i \le 0|i=\{1,2\}\}} be absolute fitness of uniparental or
% biparental offspring.
% Define \m {\tau_i = \tau'_i / \max(\tau'_1, \tau'_2)} as the rate of survival
% relative to organisms with higher fitness.
%
% \startsubsection[title={Sex before uniparental/biparental}]
% One way to choose parent(s) is to determine the sex of the seed-parent before
% deciding the number of parents.
% The following decision tree shows a sequence of events starting
% from a root.
% Each edge is labeled by probability.
%
% \starttikzpicture
% \graph[tree layout, level distance=2cm, sibling distance=2cm]
% {
%     root -> {
%         female [>"$\frac{N_f \sigma}{N_h + N_f \sigma}$"],
%         hermaprhodite [>"$\frac{N_h}{N_h + N_f \sigma}$"]},
%     female -> fb [as=biparent, >"$1$"],
%     hermaprhodite -> {
%         hu [as=uniparent, >"$a$"],
%         hb [as=biparent, >"$1-a$"]},
%     fb -> {fba [as=alive, >"$\tau_2$"], fbd [as=dead, >"$1-\tau_2$"]},
%     hu-> {hua [as=alive, >"$\tau_1$"], hud [as=dead, >"$1-\tau_1$"]},
%     hb -> {hba [as=alive, >"$\tau_2$"], hbd [as=dead, >"$1-\tau_2$"]};
% };
% \stoptikzpicture
%
% In the end, there are six possible outcomes, of which only three lead
% to viable offspring.
% The following table summarizes joint probabilities of all events and
% joint probability of the sex of seed-parent and the number of parents
% conditional on survival of offspring.
%
% \starttabulate[|c|c|c|m|m|]
% \HL
% \NC Sex \NC Number \NC Survival
% \NC {\rm Probability} \NC {\rm Probility | Alive} \NR
% \HL
% \NC Female \NC 2 \NC Alive
% \NC \frac{N_f \sigma}{N_h + N_f \sigma } \tau_2
% \NC \frac{N_f \sigma \tau_2}
% {N_h a \tau_1 + N_h (1 - a) \tau_2 + N_f \sigma \tau_2}  \NR
%
% \NC Female \NC 2 \NC Dead
% \NC \frac{N_f \sigma}{N_h + N_f \sigma} (1 - \tau_2) \NC \NR
%
% \NC Hermaphrodite \NC 1 \NC Alive
% \NC \frac{N_h}{N_h + N_f \sigma} a \tau_1
% \NC \frac{N_h a \tau_1}
% {N_h a \tau_1 + N_h (1 - a) \tau_2 + N_f \sigma \tau_2} \NR
%
% \NC Hermaphrodite \NC 1 \NC Dead
% \NC \frac{N_h}{N_h + \sigma N_f} a (1 - \tau_1) \NC \NR
%
% \NC Hermaphrodite \NC 2 \NC Alive
% \NC \frac{N_h}{N_h + \sigma N_f} (1 - a) \tau_2
% \NC \frac{N_h (1 - a) \tau_2}
% {N_h a \tau_1 + N_h (1 - a) \tau_2 + N_f \sigma \tau_2} \NR
%
% \NC Hermaprhodite \NC 2 \NC Dead
% \NC \frac{N_h}{N_h + N_f \sigma} (1 - a) (1 - \tau_2) \NC \NR
% \HL
% \stoptabulate
% \stopsubsection
%
% \startsubsection[title={Uniparental/Biparental before Sex}]
%
% Another way to choose parent(s) is to decide the number of parents before
% deciding the sex of seed-parent as depicted below.
% This sequence is currently implemented in {\sc SelfignSim}.
%
% \starttikzpicture
% \graph[tree layout, level distance=2cm, sibling distance=2cm]
% {
%     root -> {
%         uniparent [>"$\frac{N_h a}{N_h + N_f \sigma}$"],
%         biparent [>"$\frac{N_h (1 - a) + N_f \sigma}{N_h + N_f \sigma}$"]},
%     uniparent -> hu [as=hermaprhodite, >"$1$"],
%     biparent -> {
%         fb [as=female, >"$\frac{N_h(1-a)}{N_h(1-a) + N_f \sigma}$"],
%         hb [as=hermaprhodite, >"$\frac{N_f \sigma}{N_h(1-a) + N_f \sigma}$"]},
%     hu -> {hua [as=alive, >"$\tau_1$"], hud [as=dead, >"$1-\tau_1$"]},
%     fb -> {fba [as=alive, >"$\tau_2$"]], fbd [as=dead, >"$1-\tau_2$"]},
%     hb -> {hba [as=alive, >"$\tau_2$"], hbd [as=dead, >"$1-\tau_2$"]};
% };
% \stoptikzpicture
%
% Again, the following table surmarizes joint and conditonal probabilities of
% events.
%
% \starttabulate[|c|c|c|m|m|]
% \HL
% \NC Number \NC Sex \NC Survival
% \NC {\rm Probability} \NC {\rm Probability | Alive} \NR
% \HL
%
% \NC 1 \NC Hermaphrodite \NC Alive
% \NC \frac{N_h a}{N_h + N_f \sigma} \tau_1
% \NC \frac{N_h a \tau_1}
% {N_h a \tau_1 + N_h (1 - a) \tau_2 + N_f \sigma \tau_2} \NR
%
% \NC 1 \NC Hermaphrodite \NC Dead
% \NC \frac{N_h a}{N_h + N_f \sigma} (1 - \tau_1) \NC \NR
%
% \NC 2 \NC Female \NC Alive
% \NC \frac{N_f \sigma}{N_h + N_f \sigma} \tau_2
% \NC \frac{N_f \sigma \tau_2}
% {N_h a \tau_1 + N_h (1 - a) \tau_2 + N_f \sigma \tau_2} \NR
%
% \NC 2 \NC Female \NC Dead
% \NC \frac{N_f \sigma}{N_h + N_f \sigma} (1 - \tau_2) \NC \NR
%
% \NC 2 \NC Hermaphrodite \NC Alive
% \NC \frac{N_h (1 - a)}{N_h + N_f \sigma} \tau_2
% \NC \frac{N_h (1 - a) \tau_2}
% {N_h a \tau_1 + N_h (1 - a) \tau_2 + N_f \sigma \tau_2} \NR
%
% \NC 2 \NC Hermaphrodite \NC Dead
% \NC \frac{N_h (1 - a)}{N_h + N_f \sigma} (1 - \tau_2) \NC \NR
% \HL
% \stoptabulate
%
% \stopsubsection
%
% \startsubsection[title={Equivalence}]
% Comparing the last column from both tables show that both seqeunces
% of events leading to outcomes with identical probabilities.
% \stopsubsection
%
% \stopsection

\stoptext
