%&context
\setuppapersize[letter]
\setuplayout[topspace=0.75in, backspace=0.75in, height=middle, width=middle]
\setupbodyfont[12pt]
\setupwhitespace[medium]
\setuppagenumbering[location=footer]
\setupheader[state=none]

\starttext

\title{Design of a Selfing Simulator}

\startsection[title={Introduction}]
This document describes the design of \cap {Selfingsim},
a forward-in-time population genetics simulator with selfing
under pure-hermaphroditism, androdioecy, or gynodioecy.
\stopsection

\startsection[title={Design}]
One run of \cap {Selfingsim} comprises three distinct phases: initialization,
iteration, and termination.
These phases are described below.

\startsubsection[title={Initialization}]
The first stage of a simulation is to set up a population.
Given a user-specified population size \m{N}, the simulator creates \m{N} organisms.
Under pure-hermaphroditism, all organisms are hermaphrodites.
Under andro- and gynodioecy, an additional user-defined parameter
\m{N_h} such that \m{N_h < N} determines the number of hermaprhodites in the population.
The remaining \m{N - N_h} organisms are respectively male or female in andro- and gynodioecy.
Each organisms have \m{L} autosomal genes, where number of loci \m{L} is again user-sepcified.

The next step is to set initial genotype of all organisms.
Three initialization methods are currently available.
One method is to assign an identical allele to all genes.
Another method is to make all genes distinct.
The last method is to randomly set gene from equiprobable \m{M} alleles.

Primary runs in pure-hermaphroditism case revealed that staring a simulation with a monomorphic
population took long time to reach mutation-drift equilibrium, whereas
staring with a highly polymorphic poplation quickly shed excess alleles.
Therefore simulations so far have been initialized with the second method.
\stopsubsection

\startsubsection[title={Iteration}]
A simulation runs
\stopsubsection

\startsubsection[title={Termination}]
During the termination phase of a simulation, the state of the last generation is stored in file.
In addition to genotypes, the time since last outcrossing event is also recorded for all organism.
\stopsubsection

\stopsection

\stoptext
