%&context
\setuppapersize[letter]
\setuplayout[topspace=0.75in, backspace=0.75in, height=middle, width=middle]
\setupwhitespace[medium]
\setuppagenumbering[location=footer]
\setupheader[state=none]

\starttext

\title{Design of a Selfing Simulator}

\startsection[title={Introduction}]
This document describes the design of {\sc SelfingSim},
a forward-in-time population genetics simulator with selfing
under pure-hermaphroditism, androdioecy, or gynodioecy.
\stopsection

\startsection[title={Design}]
{\sc SelfingSim} simulates a constant size population without
generation overlap under the infinite-alleles model of mutation.
One run of {\sc SelfingSim} can be broken into three distinct phases:
initialization, iteration, and termination.
Explanations of each phase are provided below.

\startsubsection[title={Initialization}]
At the beginning of a simulation, {\sc SelfingSim} initializes a population
based on users' input.
The first step is to assign \m {N} reproducing orgniams with \m {L} autosomal
loci to the population.
Although all organisms are hermaphrodites under pure-hermaphroditism,
only \m {N_h} organisms are hermaprhodites under
andro- and gynodioecies.
The rest of the population under the latter two models are males or females
under andro- or gynodioecy, respectively.

The next step is to determine haplotypes of the orngiams.
\cap {Selfingsim} currently supports four methods to initialze haplotypes,
\quotation {\tt monomorphic}, \quotation {\tt unique}, \quotation {\tt count},
and \quotation {\tt frequency}.
The first two modes are straightfoward.
All genes are identical under \quotation {\tt monomorphic} or distinct under
\quotation {\tt unique}.
Under \quotation {\tt count}, a state of a gene is drawn from \m {M} alleles
equiprobably.
Finally, \quotation {\tt freqnecy} randomly determines a state of a gene
similar to \quotation {\tt count}, but a user can specify frequency of
each allele.
A common attributes of the currently implemented haplotype initialization
methods is independence of each gene.
It is impossible to assign multi-locus haplotype or genotype directly.
\stopsubsection

\startsubsection[title={Iteration}]
Immediately following the completion of the initialization,
{\sc SelfignSim} starts simulating evolution of the population
for \m {T_{max}} generations.
Additionally, \m {T_B} generations of burn-in can be preceded the main
iterations.
The burn-in allows the population to reach a equilibrium state.
The simulator performs the exactly same sequence of events
during burn-in and main iterations.

At the beginning of a generation/iteration, genes of the current constituents
of the simulation organisms are mutated.
A gene at locus \m {i} mutates to a completely new state with probability \m {m_i}.
Otherwise, a gene remains unchanged.

After mutation, the simulater initiates model-specific mating.

\startsubsubsection[title={Pure-Hermaphroditism}]
\stopsubsubsection

\startsubsubsection[title={Androdioecy}]
\stopsubsubsection

\startsubsubsection[title={Gynodioecy}]
\stopsubsubsection

\stopsubsection

\startsubsection[title={Termination}]
During the termination phase of a simulation, the state of the last generation is stored in file.
In addition to genotypes, the time since last outcrossing event is also recorded for all organism.
\stopsubsection

\stopsection

\stoptext
