\documentclass[12pt]{article}
\usepackage{amsmath}
\usepackage{algorithm}
\usepackage{algpseudocode}
\usepackage{hyperref}

% \algrenewcomment[1]{\(\triangleright\) #1}
\algnewcommand{\LineComment}[1]{\State \(\triangleright\) #1}

\begin{document}
\title{Designing Forward Simulator for Partially Selfing Population}
\date{\today}
\maketitle

\section{Synopsis}
\label{sec:synopsis}

This document describes design and implementation of a forward
simulation for studying heterozigosities of linked loci in a
partially-selfing population, which is modeled after hermaphrodite
species.
The simulation is implemented using
\href{http://simupop.sourceforge.net/}{simuPOP}.

\section{Design}
\label{sec:design}

A simulated population is modeled after a panmictic population of
hermaphroditic organisms with non-overlapping generations.
In the first part of this document
biological basis of the simulated population is described in detail.

\subsection{Population Structure}
\label{sec:population-structure}

The simulated population consists of hermaphroditic organisms
in a single deme, and its population size \(N\) remains constant
over generations.
After a single bout of mating, a parental generation immediately dies
down, and their offspring takes place of their parents.
A fraction \(S\) of the population are reproduced by selfing,
and the rest do so by outcrossing.
In particular outcrossers cannot reproduce by self-fertilization.

\subsection{Mating Scheme}
\label{sec:mating-scheme}

At mating stage of their life cycle,
\(N\) organisms participate in reproduction both via
self-fertilization or mating with other organisms.
Any organism can perform both modes of reproduction.
Selfing and outcrossing result in \(SN\) and \((1-S)N\) offspring.
In total \(N\) parents leave \(N\) offspring, and population size does
not change over generations.


\subsection{Recombination}
\label{sec:recombination}

A simulation is performed on two linked autosomal loci.
We assume that there is no intra-locus recombination in either locus,
but inter-loci recombination occurs at rate \(r\) per generation.

\subsection{Mutation}
\label{sec:mutation}

Evolution of loci is modeled under the infinite-site model of
mutations with rates \(\mu_{1}\) and \(\mu_{2}\) per generation
for the first and second loci, respectively.
Furthermore, these mutations are considered to be selectively neutral.
Mutations occurs prior to mating.  However, this condition has
very little relevance to evolution of loci.
The only place this choice affects the evolution is that organisms in
the initial generation experience mutations in addition to those in
any subsequent generations.

\section{Implementation}
\label{sec:implementation}

The simulation is implemented using
\href{http://simupop.sourceforge.net/}{simuPOP}, a forward population
genetic simulation framework available as a Python package.
This framework provides a convenient way to implement forward
simulations very quickly by providing a lot of ready-made operations
such as transmission of chromosomes with or without recombination.
Those provided operations are implemented in C++ in order to make
simulations run efficiently.

At the same time, a high level of flexibility is also provided in the
framework by allowing users to plug in additional operations written
in Python (or C++ for better run time efficiency).
For example, simuPOP does not provide an operator for hermaphroditic
mating,
but it is straight forward to implement this scheme.

From a perspective of a user, who uses siumPOP to implement forward
simulations, one concern about this package is that it
is sometimes hard to make sure that simulations are implemented as
designed.
This is due to difficulties of understanding how various operators
interact with each other under complex scenarios.

\subsection{Algorithm}
\label{sec:algorithm}

Having briefly gone over characteristics of simuPOP, simulation
algorithms for inbreeding.py is described here.
An overall structure of the simulation is given in code \ref{algo:overall}.
In words, the simulation goes by alternatively performing
mutation and reproduction for a user specified number of generations
after initializing population.
In the current simulation, initialization of population is done by
simuPOP-provided function.

Mutations under the infinite-sites model is approximated by
\(n\)-sites model with a large \(n\).
When a new mutation event occurs in a population, one of the unused
sites are assigned to the mutation.
Absence and presence of the mutation are coded as binary variable;
zero is absent, one is present.
All of \(n\) sites are used up after \(n\)-th mutation.
When this situation arises, simulator tries to free up some sites,
which holds information about a mutation either lost from or fixed in
the population due to genetic drift.
If no site becomes available after the recycle, an error is reported.
Additionally, a simulation is aborted.
Dynamically increasing \(n\) during simulation is considered, but it
has not been implemented yet.
The current implementation uses \(n = 256\), and it is illustrated in
code \ref{algo:mutation}.

For this simulation, simuPOP-provided operations are sufficient for
all events except mutations and selecting pairs of parents in
outcrossing reproduction.

\subsubsection{Initialization}
\label{sec:initialization}

At initialization, a population is filled with
\(N\) original organisms.
All of these organisms carry ancestral alleles at both loci.
Hence, the population is monomorphic.

\subsubsection{Iterations}
\label{sec:iterations}

After the population is initialized, the simulation moves forward by
alternatively applying mutation and reproduction to the population for
\(ngen\) generations.

Occurrence of mutations at the first and second locus follows
Bernoulli distribution with parameter \(\mu_{1}\) and \(\mu_{2}\).
For each individual, both genes at each locus are tested for the
presence of mutation.
When a mutation is present in an organism, an appropriate gene gets
a novel allele by assigning an unused site and by setting the value at
the site to `1' for the organism and `0' in other organisms.

After tests of new mutations are performed, reproduction takes place.
All organisms have an equal chance of reproduction.
A fraction \(S\) of offspring is generated by selfing, which is
implemented by randomly picking an organism from the parental
population.
After a parent is selected, single offspring is created.
This procedure is repeated until the total number of children reaches
\(S N\).

Outcrossing follow a similar procedure.
A random pair of organisms from the current generation are chosen as
parents, and the pair create a single offspring.
Again, these steps are repeated until the total number of children from
the outcrossing reaches \((1 - S)N\) (code \ref{algo:outcrossing}).
This scheme implies that, even if a parent has multiple offspring,
another parent of its offspring are not necessarily the same
organism.
In both mating schemes, recombination occurs at a rate \(r\).

\subsubsection{termination}
\label{sec:termination}

After iterating for \(ngen\) generations, we have a final population,
the fraction of heterozygotes among the population is recorded.
Then, a run of the simulation is terminated.

\begin{algorithm}
  \caption{Overall structure}
  \label{algo:overall}
  \begin{algorithmic}
    \State Initialize \(N\) \Comment {population size}
    \State Initialize \([\mu_{1}, \mu_{2}]\) \Comment {mutation rate
      at loci 1 and 2}
    \State Initialize \(r\) \Comment {recombination rate}
    \State Initialize \(S\) \Comment {selfing rate}
    \State Initialize \(ngen\) \Comment {number of generations}
    \State Initialize \(nrep\) \Comment {number of replications}

    \ForAll {\(nrep\)}

    \LineComment {initialize population}
    \State \(pop \gets\) create population of size \(N\)

    \LineComment {Iteration}
    \For {\(ngen\) generations}

    \State \(pop \gets\) \Call{Add mutations}{$pop$}

    \LineComment {reproduction}
    \LineComment {\(pop'_{s}, pop'_{o}\) are offspring of selfers and outcrossers}
    \State \(pop'_{s} \gets\) \Call{Selfing}{$pop, r$}
    \State \(pop'_{o} \gets\) \Call{Outcrossing}{$pop, r$}
    \LineComment {offspring generation replaces parental generation}
    \State \(pop \gets pop'_{s} \cup pop'_{o}\)
    \EndFor

    \State Compute heterozygosity from \(pop\)

    \EndFor

  \end{algorithmic}
\end{algorithm}

\begin{algorithm}
  \caption{Mutation}
  \label{algo:mutation}
  \begin{algorithmic}
    % need to use $$ for inline math to workaround a known bug in
    % ifthen package
    \Function{Add mutations}{$pop$}
    \For {\(i = 1 \to N\)}
    \For {\(j = 1 \to 2\)} \Comment {iterate over locus}
    \For {\(k = 1 \to 2\)} \Comment {iterate over chromosome}
    \If {\(rng < S\)}
    \If {No unused site}
    \State Recycle monomorphic sites
    \If {No unused site}
    \State Print error and abort
    \EndIf
    \EndIf
    \State Select an unused site \(sites_{l}\)
    \State \(pop.individual[i].chromosome[k].sites[l] \gets 1\)
    \EndIf
    \EndFor
    \EndFor
    \EndFor
    \State \Return \(pop\)
    \EndFunction
  \end{algorithmic}
\end{algorithm}

\begin{algorithm}
  \caption{Selfing}
  \label{algo:selfing}
  \begin{algorithmic}
    \Function{Selfing}{$pop, r$}
    \State Initialize \(pop_{s}'\) as an empty list
    \For {\(i = 1 \to SN \)}
    \State \(parent \gets\) pick an organism from \(pop\)
    \State \(child \gets\) produced from \(parent\) with recombination
    rate \(r\)
    \State Extend \(pop_{s}'\) by adding \(child\)
    \EndFor
    \State \Return \(pop_{s}'\)
    \EndFunction
  \end{algorithmic}
\end{algorithm}

\begin{algorithm}
  \caption{Outcrossing}
  \label{algo:outcrossing}
  \begin{algorithmic}
    \Function{Outcrossing}{$pop, r$}
    \State Initialize \(pop_{o}'\) as an empty list
    \For {\(i = 1 \to (1 - S)N\)}
    \State \(parents \gets\) pick a unique pair from \(pop\)
    \State \(child \gets\) produced from \(parents\) with
    recombination rate \(r\)
    \State Extend \(pop_{o}'\) by adding \(child\)
    \EndFor
    \State \Return \(pop_{o}'\)
    \EndFunction
  \end{algorithmic}
\end{algorithm}


\end{document}
%%% Local Variables:
%%% mode: latex
%%% TeX-master: t
%%% End:
